\documentclass[8pt, letterpaper]{article}
\usepackage[utf8]{inputenc}
\usepackage[spanish,es-nodecimaldot]{babel}
\usepackage[top=1in, bottom=1in, left=1in, right=1in]{geometry}

\usepackage{hyperref}
\usepackage{svg}
\usepackage{csquotes}
\usepackage{graphicx}

\usepackage{amsmath}
\usepackage{amsthm}
\usepackage{amssymb}

\usepackage{import}

\title{%
  Tarea 1\\
  {\large{Información General}}}
\author{Diego Méndez Medina}
\date{}
\begin{document}
\ttfamily
\maketitle
\rmfamily
\begin{enumerate}
\item Considera un sistema distribuido con $n \geq 2$ procesos
  $p_1, p_2, ..., p_n$, en el que la gráfica de comunicación es la completa
  $K_n$. El sistema es síncrono pero la comunicación \textit{no es confiable};
  sea $P$ el conjunto de todos los procesos que envían mensajes en el tiempo
  $d$; entonces, hay dos posibilidades, todos los mensajes de $P$ llegan a su
  destino en el tiempo $d+1$, o uno de ellos se pierde y nunca llega a su destino
  y los otros en $P$ si llegan en el tiempo $d+1$.

  Considera un algoritmo $A$ en el que cada proceso $p_i$ tiene como entrada un
  identificador $ID_i$, que es un número natural(diferente al de los demás),
  y cada proceso $p_i$ simplemente envia su $ID$, a los otros $n-1$ procesos.
  Dibuja cuales son todos los estados \textit{globales} posibles(mundos posibles) en el tiempo 1 (los procesos mandan sus mensjaes en el tiempo 0). En cada estado global, especifica el estado local de cada proceso, es decir, la información
  que cada proceso tiene en estado global; y entre cada par de estados globales pinta una arista con los procesos que no pueden \textit{distinguir} entre esos 
  estados. ¿Es posible que cada proceso elija consistentemente uno de los $IDs$
  de entre los que recibió de forma tal que en cada estado global todos los
  procesos eligen el mismo $ID$? Argumenta tu respuesta.
  \ttfamily  
  %Respoesta

  \hfill\break
  {\bf Solución:}\hfill\break
  Primero un poco de notación. Siguiendo la notiación de la ayudantía, la
  siguiente imagen representa el proceso $n$ en la gráfica y los números azules
  son su estado local, es decir, los procesos que se comunicaron con él.
  %ejemplo
  \begin{figure}[h]
    \centering
    \def\svgwidth{2cm}
    \import{src/}{ejemplo.pdf_tex}
  \end{figure}
  \hfill\break
  %definimos n=3 y mostramos la grafica en tiempo cero
  Tomemos $n=3$. Enumeramos los tres procesos comenzando desde el cero,
  así tenemos el conjunto $P=\{p_0, p_1, p_2\}$, que de acuerdo al algoritmo
  descrito cada proceso $p_i$ envía su $ID$ al resto de procesos. 

  La siguiente gráfica representa el tiempo cero.
  \begin{figure}[h]
    \centering
    \def\svgwidth{5cm}
    \import{src/}{0.pdf_tex}
  \end{figure}
  \hfill\break
  %Mundos en tiempo uno
  Lo siguiente es mostrar los casos globales posibles en el tiempo {\bf uno}.
  El primer caso que mostraremos es el \textit{perfecto}, cuando cada
  $p_i$ en $P$ recibe el $ID$ del resto de los procesos.
  \begin{figure}[h]
    \centering
    \def\svgwidth{5cm}
    \import{src/}{mundo_perfecto.pdf_tex}
  \end{figure}

  %que pex con la pregunta
  Ahora, de acuerdo con la descripción del problema existen dos posibilidades,
  la primera antes mencionada y la segunda: ``uno de los mensajes de $P$ se
  pierde y nunca llega a sus destino y los otros en $P$ si llegan en
  tiempo $d+1$''.

  %Saliendo del problema
  $P$ es un conjunto de procesos no de mensajes, pero entendemos que lo que
  en realidad se pierde es un mensaje de algún proceso $p_i$, no todo el
  proceso. Por ejemplo, está el caso donde todos los mensajes de $p_1$ y
  $p_2$ llegan a sus destinos, pero se perdio alguno de $p_0$, digamos el
  que tenía como destino $p_1$, entonces el mensaje de $p_0$ a $p_2$ sí
  llegó. La siguiente gráfica muestra dicho caso.
  \begin{figure}[h]
    \centering
    \def\svgwidth{5cm}
    \import{src/}{falla.pdf_tex}
  \end{figure}

  La flecha roja no indica que la comunicación se haya roto, simplemente es para
  ver graficamente cual fue el mensaje no llegó.

  Viendo el ejemplo anterior se observa que cada arista puede fallar dos veces,
  así el numero de casos a mostrar donde falla es $2\cdot |E| = 2 \cdot 3 = 6$. 
  \newpage
  %Casos globales juntos
  A continuación se muestran esos seis casos, junto al caso perfecto, añadiendo
  el requisito de la arista entre par de estados globales que no sean capaces
  de distinguir.
  \begin{figure}[h]
    \centering
    \def\svgwidth{15cm}
    \import{src/}{globales.pdf_tex}
  \end{figure}
  \hfill\break
  Se observa que para cada proceso cinco, donde para cada proceso uno de estos
  cinco es el estado global perfecto, de los siete estados globales es el
  mismo. Al solo conocer su estado local es imposible para el proceso saber
  que pasa con los demás procesos, ya sea que este en el caso perfecto o en
  el que un mensaje no llegó.

  %Conclusión
  Por lo antes mencionado, dada la descripción del problema es imposible que
  los tres procesos escojan algún $ID$, si bien es cierto que existe el estado
  global donde no ocurren errores no sabemos si es el más frecuente o no;
  peor aún si todos los estados globales tienen la misma probabilidad de ocurrir
  y el proceso donde se ejecute el algoritmo, $p_i$, tiene como estado local
  la lista con los otros dos estados la probabilidad de que los demás no
  tengan la misma información que el proceso $p_i$ es de $.8$. Por otro lado,
  podriamos esperar a tomar la decisión, el algoritmo mencionado no indica que
  se pueda hacer eso pero si nos lo planteamos  salen otras cuestiones:
  si el estado $p_i$ solo recibe un $ID$ lo sensato seria que todos escojan
  ese, pero ahora surge la duda ¿De que manera les comunica que solo recibio
  uno?, ¿Qué tal si su mensaje indicando que solo recibio un $ID$ no llega a
  todos?.

  Resumiendo: En un sistema donde la comunicación no es confiable la decisión
  de escojer el mismo $ID$ consistente con los otros dos procesos no es
  posible, por que no hay forma de asegurar que información tienen los demás.
  \rmfamily
  \newpage
  % Pregunta dos 
\item  Retomando el proceso de los dos enamorados con los mismos requerimientos
  vistos en clase, responda las siguientes preguntas:
  \begin{itemize}
  \item Suponga que las citas sólo se pueden realizar entre las \textit{21:00}
    y las \textit{22:00}. ¿Tiene solución el problema en este caso?
    
    \hfill\break
    %Respoesta
    {\bf Solución:}\hfill\break
    No, lo que hace es disminuir el margen de error. Ejemplo:

    Antes el enamorado podia decirle que a las \textit{10:00}, ella le
    confirmaba, pero despúes le surge un imprevisto y le pide que mejor sea a
    las \textit{14:00}, ella no obtiene respuesta. El ultimo mensaje forma
    parte del conjunto con el mínimo de mensajes para la solución al problema,
    es decir si el mensaje llega ambos se ven a las \textit{14:00 hrs},
    si no él llegará a las diez y ella a las catorce. Para ambos no hay forma
    de actuar estando seguros de lo que hara o lo que conoce el otro.
    ¿Qué tal que la/el otra/otro envio un mensaje que no me llegó?

    Con el planteamiento de este inciso puede que el le pida sea a las
    \textit{21:05} ella confirma pero recuerda que no tiene permiso hasta
    las \textit{21:45}, supongamos que, de nuevo, no obtiene respuesta.
    Termina ocurriendo lo mismo.

    El problema no radica en el rango de hora, si no en que el sistema de
    comunicación no es confiable, pues algún mensaje puede perderse en
    el camino.
    \hfill\break
  \item ¿Se puede resolver el problema de los dos amantes cuando se establece
    como requerimiento que, los amantes deben ser capaces de coordinar una
    hora para reunirse sólo cuando ningún mensaje se pierde, y en cualquier
    otro caso, ninguno de ellos debería presentarse?
    
    \hfill\break
    %Respoesta
    {\bf Solución:}\hfill\break
    Si ellos son conscientes que hay mensajes perdidios puede que sí.
    Para empezar el requerimiento nos asegura que en caso de no haber
    mensajes perdidos ambos son capaces de coordinar una hora, es decir
    no hay fallas en la comunicación y si las llegase a haber, asumiendo que
    ellos lo saben, puede que los mensajes que no se enviaron no formen parte
    de la solución minima, llamemosles mensajes basura. Así si ocurre una falla
    y todos los mensajes perdidos son mensajes basura ambos tenían la
    información necesaria para cenar juntos con lo que el fallo no repercutiria
    pero ellos al no saber mejor deciden
    no ir, por otro lado, si no todos los mensajes perdidos son basuras entonces
    al menos un mensaje es necesario para acordar la reunión; de nuevo asumimos
    que ellos saben que se perdio algún mensaje entonces no van, esta vez
    acertando pues el encuentro no se haría.

    \hfill\break
    La respuesta asumiendo que ambos son conscientes de que existe perdida de
    mensajes es \textit{puede que sí} por que en el caso perfecto donde no hay
    perdida de mensajes, ambos lo saben entonces se reunen, {\bf pero} al
    no ser un sistema confiable puede que este caso nunca se presente y no
    se lleve acabo la cena, no logrando el cometido. Con lo que en realidad
    la respuesta es no.
    
    \hfill\break
    Si ellos no son conscientes que hay mensajes perdidos la respuesta es
    definitivamente no pues nos encontramos en la misma situación que sin este
    requerimiento. Al no ser un sistema que garantice toda comunicación no
    pueden estar seguros si no se envio o recibio algún mensaje.
    \hfill\break
  \item Consideremos una variación: Los dos amantes se han dado cuenta de que no
    necesitan ponerse de acuerdo sobre una hora exacta para la reunión, está
    bien si sus horas de reunión son lo suficientemente cercanas.
    En otras palabras, cada uno deberia eventualmente elegir un tiempo,
    de modo que los dos tiempos estén lo suficientemente cerca. ¿Se puede
    resolver su problema?
    \newpage
    \hfill\break
        {\bf Solución:}
        
    La respuesta vuelve a ser no. Primero pense:
    
    \hfill\break
    \textit{El problema nunca especifica cuando dos horas son suficientementes
      cercanas, entonces podria decir que dos horas son suficientemente cercanas
      en un rango de un día (24 hrs).}

    \hfill\break
    Pero luego me dí cuenta que eso no tiene sentido en la vida real, en el
    modo que alguien no va a esperar un día en un restaurante, y además
    que el sistema no sea confiable vuelve a ser un problema. Al pensar que
    dos horas suficientementes cercanas son de un día o hasta una semana solo
    hace lo contrario que el primer inciso de la pregunta dos, ampliar el margen
    de error.
    
    \hfill\break
    Supongamos que un día es el tiempo suficientemente cerca para
    que se vean, ahora esta el escenario donde ella le pide se vean a las doce
    horas del lunes, el confirma pero despues pide sea a las trece horas
    del mismo lunes, ella recibe el mensaje. Al estar en el rango de las
    veinticuatro horas ella sigue sin cambiar de hora, está dispuesta a
    esperar una hora. Pero ahora ella decide cambiar la hora a
    las nueve de la mañana del miercoles, solo que el mensaje se pierde,
    ya no entra en el rango de las veinticuatro horas, él debe cambiar su hora. 
    El cree que sigue en el rango así que no la cambia.
    Él llega el lunes a las trece, espera hasta las trece del martes y ella no
    llegá. Misión fallida.
    \hfill\break
  \end{itemize}
  \ttfamily
  %Respoesta
  \rmfamily
\item Investigue y explica brevemente el protocolo TCP. ¿Es posible resolver
  el problema de los dos amantes si hay un canal TCP confiable entre ambos
  amantes?
  \ttfamily
  
  \hfill\break
  %Respoesta
      {\bf Solución:}
      \hfill\break

      Primero hay que entender que TCP, protocolo de control de transmisión, es
      un modelo que busca estandarizar la conexión de computadoras.
      Originalmente contaba de cuatro capas:
      \begin{enumerate}
      \item[4.]\textit{Aplicación}
      \item[3.] \textit{Transporte}
      \item[2.] \textit{Internet}
      \item[1.] \textit{Enlace}
      \end{enumerate}
      Los cuales como podemos ver estan enumerados de abajo hacia arriba, aunque
      en realidad la dirección depende si estas recibiendo o enviando
      tráfico(datos).

      \hfill\break
      El protocolo se actualizo, ahora cuenta con una capa extra y renombraron
      la capa internet. Aquí la nueva lista y quienes la conforman:
      \begin{enumerate}
      \item[5.]\textit{Aplicación}
        \hfill\break
        En esta capa se encuentran los protocolos de las aplicaciónes como
        lo son HTTP, FTP y SMTP.
      \item[4.] \textit{Transporte}
        Aquí se encuentran los protocolos de transporte, los más comunes son
        TCP y UDP. Tambíen se encuentran los puertos.
      \item[3.] \textit{Red}
        Se encuentra el protocolo de itnernet (IP) y los enrutadores.
      \item[2.] \textit{Enlace}
        Ethernet y switches.
      \item[1.]\textit{Fisica}
        Cables, dispositivos con adaptadored de red.
      \end{enumerate}
      Como sirven las capas al enviar información:
      
      Cuando alguien envia información está comienza en la capa cinco y va
      bajando, cada capa agrega su propia informiación en bits, pasar por las
      cinco capas se llama \textit{encapsulación}. En cuanto alcanzamos la
      capa física, la información es transmitida a través del dispositivo del
      receptor. Breve descripción del encapsulamiento:
      \begin{enumerate}
      \item[5.]\textit{Aplicación}
        \hfill\break
        En está capa se recibe la información.
        
        El paquete se ve así:
        
        \hfill\break
        [DATA] y simplemente se le llama data(información).
      \item[4.] \textit{Transporte}
        \hfill\break
        La información de transporte se añade, en esté caso con un encabezado
        TCP. Cada vez que un encabezado se agrega esté contiene información
        específica. Los encabezados TCP tienen la información de origen, destino
        , números de puerto, secuencias de numero y algunos bits más de
        información.
        
        \hfill\break
        Aqui la información se ve así:
        \hfill\break
        [TCP][DATA]
        y se llama segmento
      \item[3.] \textit{Red}
        \hfill\break
        Se agrega el encabezado IP, que contiene las direcciones IP de origen
        y destino y más bits de información.
        
        \hfill\break
        La información se ve así:
        \hfill\break
        [IP][TCP][DATA]
        Y se llama Paquete
      \item[2.] \textit{Enlace}
        \hfill\break
        Aquí se agrega un encabezado con las direcciónes MAC de origen y destino
        más y un \textit{traíler} que es información adicional que se agrega
        al inicio y al final de un bloque de información que se va a emitir,
        sirve para marcar el final de bloque y para manejar su información,
        este en partícular lleva información para que el receptor pueda
        hacer manejo de errores y asegurarse de recibir bien la información. 

        \hfill\break
        La información se ve así:
        \hfill\break
        [ETHERNET][IP][TCP][DATA][ETHERNET]
        Y se llama Frame.
      \item[1.]\textit{Fisica}
        \hfill\break
        Se transmite físicamente.
      \end{enumerate}
      
      El dispositivo receptor comienza a \textit{decapsular} la información,
      checa la dirección MAC de destino del frame, si el frame va dirigido
      a nuestro computador o dispositivo entonces sigue subiendo.
      Ahora el computador checa si la dirección IP del paquete está dirigida
      a el, si es así sigue subiendo.
      Se lee la información de transporte y la información de la aplicación se
      envía a la aplicación receptora(WhatsApp, Browser, etc).

      \hfill\break
      De acuerdo a lo entendido y a una de mis
      fuentes\footnote{https://www.ibm.com/docs/en/aix/7.1?topic=protocols-transmission-control-protocol}
      TCP es un medio confiable para la recepción de datos, en el sentido
      que puedes saber quien la envía, si es para ti e incluso maneja errores
      para verificar que la información no este dañada, perdida, duplicada o
      entregada sin haberla pedido.

      \hfill\break
      Lamentablemente el protocolo no arregla nuestro problema, de hecho
      en algún sentido lo habiamos dado como un hecho. El protocolo es lo que
      permite que haya la comunicación entre enamorada y enamorado, nunca
      especificamos que se comunicaran por cartas o mediante una persona,
      al menos yo siempre creí que era a través de dispotivos conectados
      a internet, que para funciónar necesitan algun procotolo(TCP).

      \hfill\break
      Lo que TCP garantiza es que los mensajes que envie la enamorado no
      lleguen a otra  persona por accidente y saber cuales de los que recibe
      son de su enamorado, tambíen garantiza que el mensaje de su enamorado
      no sea interceptado por un tercero y modificado (Esto ultimo tambien
      depende un poco de la aplicación que se utilice, pero TCP implementa
      un pequeño sistema de seguridad), por último TCP asegura que el mensaje
      recibido sea el mismo al enviado.\footnote{https://cs.stanford.edu/people/eroberts/courses/soco/projects/1999-00/internet/tcp.html}

      \hfill\break
      TCP únicamente garantiza que intentara un numero indeterminado de veces
      la entrega de información, y que los frames seran entregados, en caso
      de encontrar al receptor, en el mismo orden en el que fueron enviados.
      TCP no influye en la decisión ni en los inconvenientes que tengan los
      al menos dos miembros de la comunicación, como es que se caiga la
      conexión, perdida de dispositivos, etc. El problema de los enamorados
      radica más bien en el problema de la aplicación, no de el protocolo.
\end{enumerate}
\end{document}
