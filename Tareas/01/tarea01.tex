\documentclass[8pt, letterpaper]{article}
\usepackage[utf8]{inputenc}
\usepackage[spanish,es-nodecimaldot]{babel}
\usepackage[top=1in, bottom=1in, left=1in, right=1in]{geometry}

\usepackage{hyperref}
\usepackage{svg}
\usepackage{csquotes}
\usepackage{graphicx}

\usepackage{amsmath}
\usepackage{amsthm}
\usepackage{amssymb}

\usepackage{import}

\title{%
  Tarea 1\\
  {\large{Información General}}}
\author{Diego Méndez Medina}
%\date{}
\begin{document}
\ttfamily
\maketitle
\rmfamily
\begin{enumerate}
\item Considera un sistema distribuido con $n \geq 2$ procesos
  $p_1, p_2, ..., p_n$, en el que la gráfica de comunicación es la completa
  $K_n$. El sistema es síncrono pero la comunicación \textit{no es confiable};
  sea $P$ el conjunto de todos los procesos que envían mensajes en el tiempo
  $d$; entonces, hay dos posibilidades, todos los mensajes de $P$ llegan a su
  destino en el tiempo $d+1$, o uno de ellos se pierde y nunca llega a su destino
  y los otros en $P$ si llegan en el tiempo $d+1$.

  Considera un algoritmo $A$ en el que cada proceso $p_i$ tiene como entrada un
  identificador $ID_i$, que es un número natural(diferente al de los demás),
  y cada proceso $p_i$ simplemente envia su $ID$, a los otros $n-1$ procesos.
  Dibuja cuales son todos los estados \textit{globales} posibles(mundos posibles) en el tiempo 1 (los procesos mandan sus mensjaes en el tiempo 0). En cada estado global, especifica el estado local de cada proceso, es decir, la información
  que cada proceso tiene en estado global; y entre cada par de estados globales pinta una arista con los procesos que no pueden \textit{distinguir} entre esos 
  estados. ¿Es posible que cada proceso elija consistentemente uno de los $IDs$
  de entre los que recibió de forma tal que en cada estado global todos los
  procesos eligen el mismo $ID$? Argumenta tu respuesta.
  \ttfamily  
  %Respoesta

  \hfill\break
  {\bf Solución:}\hfill\break
  Primero un poco de notación. Siguiendo la notiación de la ayudantía, la
  siguiente imagen representa el proceso $n$ en la gráfica y los números azules
  son su estado local, es decir, los procesos que se comunicaron con él.
  %ejemplo
  \begin{figure}[h]
    \centering
    \def\svgwidth{2cm}
    \import{src/}{ejemplo.pdf_tex}
  \end{figure}
  \hfill\break
  %definimos n=3 y mostramos la grafica en tiempo cero
  Tomemos $n=3$. Enumeramos los tres procesos comenzando desde el cero,
  así tenemos el conjunto $P=\{p_0, p_1, p_2\}$, que de acuerdo al algoritmo
  descrito cada proceso $p_i$ envía su $ID$ al resto de procesos. 

  La siguiente gráfica representa el tiempo cero.
  \begin{figure}[h]
    \centering
    \def\svgwidth{5cm}
    \import{src/}{0.pdf_tex}
  \end{figure}
  \hfill\break
  %Mundos en tiempo uno
  Lo siguiente es mostrar los casos globales posibles en el tiempo {\bf uno}.
  El primer caso que mostraremos es el \textit{perfecto}, cuando cada
  $p_i$ en $P$ recibe el $ID$ del resto de los procesos.
  \begin{figure}[h]
    \centering
    \def\svgwidth{5cm}
    \import{src/}{mundo_perfecto.pdf_tex}
  \end{figure}

  %que pex con la pregunta
  Ahora, de acuerdo con la descripción del problema existen dos posibilidades,
  la primera antes mencionada y la segunda: ``uno de los mensajes de $P$ se
  pierde y nunca llega a sus destino y los otros en $P$ si llegan en
  tiempo $d+1$''.

  %Saliendo del problema
  $P$ es un conjunto de procesos no de mensajes, pero entendemos que lo que
  en realidad se pierde es un mensaje de algún proceso $p_i$, no todo el
  proceso. Por ejemplo, está el caso donde todos los mensajes de $p_1$ y
  $p_2$ llegan a sus destinos, pero se perdio alguno de $p_0$, digamos el
  que tenía como destino $p_1$, entonces el mensaje de $p_0$ a $p_2$ sí
  llegó. La siguiente gráfica muestra dicho caso.
  \begin{figure}[h]
    \centering
    \def\svgwidth{5cm}
    \import{src/}{falla.pdf_tex}
  \end{figure}

  La flecha roja no indica que la comunicación se haya roto, simplemente es para
  ver graficamente cual fue el mensaje no llegó.

  Viendo el ejemplo anterior se observa que cada arista puede fallar dos veces,
  así el numero de casos a mostrar donde falla es $2\cdot |E| = 2 \cdot 3 = 6$. 
  \newpage
  %Casos globales juntos
  A continuación se muestran esos seis casos, junto al caso perfecto, añadiendo
  el requisito de la arista entre par de estados globales que no sean capaces
  de distinguir.
  \begin{figure}[h]
    \centering
    \def\svgwidth{15cm}
    \import{src/}{globales.pdf_tex}
  \end{figure}
  \hfill\break
  Se observa que para cada proceso cinco, donde para cada proceso uno de estos
  cinco es el estado global perfecto, de los siete estados globales es el
  mismo. Al solo conocer su estado local es imposible para el proceso saber
  que pasa con los demás procesos, ya sea que este en el caso perfecto o en
  el que un mensaje no llegó.

  %Conclusión
  Por lo antes mencionado, dada la descripción del problema es imposible que
  los tres procesos escojan algún $ID$, si bien es cierto que existe el estado
  global donde no ocurren errores no sabemos si es el más frecuente o no;
  peor aún si todos los estados globales tienen la misma probabilidad de ocurrir
  y el proceso donde se ejecute el algoritmo, $p_i$, tiene como estado local
  la lista con los otros dos estados la probabilidad de que los demás no
  tengan la misma información que el proceso $p_i$ es de $.8$. Por otro lado,
  podriamos esperar a tomar la decisión, el algoritmo mencionado no indica que
  se pueda hacer eso pero si nos lo planteamos  salen otras cuestiones:
  si el estado $p_i$ solo recibe un $ID$ lo sensato seria que todos escojan
  ese, pero ahora surge la duda ¿De que manera les comunica que solo recibio
  uno?, ¿Qué tal si su mensaje indicando que solo recibio un $ID$ no llega a
  todos?.

  Resumiendo: En un sistema donde la comunicación no es confiable la decisión
  de escojer el mismo $ID$ consistente con los otros dos procesos no es
  posible, por que no hay forma de asegurar que información tienen los demás.
  \rmfamily
  % Pregunta dos 
  \newpage

\item  Retomando el proceso de los dos enamorados con los mismos requerimientos
  vistos en clase, responda las siguientes preguntas:
  \begin{itemize}
  \item Suponga que las citas sólo se pueden realizar entre las \textit{21:00}
    y las \textit{22:00}. ¿Tiene solución el problema en este caso?
  \item ¿Es el problema soluble cuando se requiere que los amantes deben
    ser capaces de coordinar una hora para una cita sólo cuando ningún mensaje
    se pierde, y, en cualquier otro caso, ellos no deberian presentarse?
  \item Consideremos una variación: Los dos amantes se han dado cuenta de que no
    necesitan ponerse de acuerdo sobre una hora exacta para la reunión, está
    bien si sus horas de reunión son lo suficientemente cercanas.
    En otras palabras, cada uno deberia eventualmente elegir un tiempo,
    de modo que los dos tiempos estén lo suficientemente cerca. ¿Se puede
    resolver su problema?
  \end{itemize}
  \ttfamily
  %Respoesta
  \rmfamily
\item Investigue y explica brevemente el protocolo TCP. ¿Es posible resolver
  el problema de los dos amantes si hay un canal TCP confiable entre ambos
  amantes?
  \ttfamily
  % Respoesta 
  
  \end{enumerate}
\end{document}
