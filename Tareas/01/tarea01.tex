\documentclass[8pt, letterpaper]{article}
\usepackage[utf8]{inputenc}
\usepackage[spanish,es-nodecimaldot]{babel}
\usepackage[top=1in, bottom=1in, left=1in, right=1in]{geometry}

\usepackage{hyperref}
\usepackage{svg}
\usepackage{csquotes}
\usepackage{multicol}

\usepackage{amsmath}
\usepackage{amsthm}
\usepackage{amssymb}
\usepackage{listings}

\title{%
  Tarea 1\\
  {\large{Información General}}}
\author{Diego Méndez Medina}
%\date{}
\begin{document}
\ttfamily
\maketitle
\rmfamily
\begin{enumerate}
\item Considera un sistema distribuido con $n \geq 2$ procesos
  $p_1, p_2, ..., p_n$, en el que la gráfica de comunicación es la completa
  $K_n$. El sistema es síncrono pero la comunicación \textit{no es confiable};
  sea $P$ el conjunto de todos los procesos que envían mensajes en el tiempo
  $d$; entonces, hay dos posibilidades, todos los mensajes de $P$ llegan a su
  destino en el tiempo $d+1$, o uno dellos se pierde y nunca llega a su destino
  y los otros en $P$ si llegan en el tiempo $d+1$.

  Considera un algoritmo $A$ en el que cada proceso $p_i$ tiene com entrada un
  identificador $ID_i$, que es un número natural(diferente al de los demás),
  y cada proceso $p_i$ simplemente envia su $ID$, a los otros $n-1$ procesos.
  Dibuja cuales son todos los estados \textit{globales} posibles(mundos posibles) en el tiempo 1 (los procesos mandan sus mensjaes en el tiempo 0). En cada estado global, especifica el estado local de cada proceso, es decir, la información
  que cada proceso tiene en estado global; y entre cada par de estados globales pinta una arista con los procesos que no pueden \textit{distinguir} entre esos 
  estados. ¿Es posible que cada proceso elija consistentemente uno de los $IDs$
  de entre los que recibió de forma tal que en cada estado global todos los
  procesos eligen el mismo $ID$? Argumenta tu respuesta.
  \ttfamily
  %Respoesta
  
  \rmfamily
  % Pregunta dos 
  \newpage
\item Retomando el proceso de los dos enamorados con los mismos requerimientos
  vistos en clase, responda las siguientes preguntas:
  \begin{itemize}
  \item Suponga que las citas sólo se pueden realizar entre las \textit{21:00}
    y las \textit{22:00}. ¿Tiene solución el problema en este caso?
  \item ¿Es el problema soluble cuando se requiere que los amantes deben
    ser capaces de coordinar una hora para una cita sólo cuando ningún mensaje
    se pierde, y, en cualquier otro caso, ellos no deberian presentarse?
  \item Consideremos una variación: Los dos amantes se han dado cuenta de que no
    necesitan ponerse de acuerdo sobre una hora exacta para la reunión, está
    bien si sus horas de reunión son lo suficientemente cercanas.
    En otras palabras, cada uno deberia eventualmente elegir un tiempo,
    de modo que los dos tiempos estén lo suficientemente cerca. ¿Se puede
    resolver su problema?
  \end{itemize}
  \ttfamily
  %Respoesta
  
  \rmfamily
\item Investigue y explica brevemente el protocolo TCP. ¿Es posible resolver
  el problema de los dos amantes si hay un canal TCP confiable entre ambos
  amantes?
  \ttfamily
  % Respoesta 
  
  \end{enumerate}
\end{document}
