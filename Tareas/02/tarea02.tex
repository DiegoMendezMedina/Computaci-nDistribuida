\documentclass[8pt, letterpaper]{article}
\usepackage[utf8]{inputenc}
\usepackage[spanish,es-nodecimaldot]{babel}
\usepackage[top=1in, bottom=1in, left=1in, right=1in]{geometry}

\usepackage{hyperref}
\usepackage{svg}
\usepackage{csquotes}
\usepackage{graphicx}

\usepackage{amsmath}
\usepackage{amsthm}
\usepackage{amssymb}

\usepackage{import}

\title{%
  Tarea 2\\
  {\large{}}}
\author{Diego Méndez Medina}
\date{}
\begin{document}
\ttfamily
\maketitle
\rmfamily
\begin{enumerate}
 %01
\item Considera un sistema distribuido representado como una gráfica de tipo
  anillo, cuyos canales son bidireccionales, con $n = mk$ procesos, con $m >1$
  y $k$ impar. Los procesos en las posiciones $0, k, 2k, ..., (m-1)k$ son
  macados inicialmente como líderes, mientras que procesos en otras posiciones
  son seguidores. Todos los procesos tienen un sentido de dirección y pueden
  distinguir su vecino izquierdo de su vecino derecho, pero ellos no tienen
  información alguna acerca de sus $ids$.

  El algoritmo $1$ está destinado a permitir que los lideres recluten
  seguidores. No es difícil ver que todo seguidor eventualmente se agrega a sí
  mismo a un árbol enraízado con padre en algún líder. Nos gustaría que todos
  esos árboles tuvieran aproximadamente el mismo número de nodos.
  \begin{itemize}
  \item ¿Cuál es el tamaño mínimo y máximo posible de un árbol?
  \item Dibuja el resultado de una ejecución para e algoritmo con $k = 5$ y
    $m = 4$
  \end{itemize}
  % Algoritmo 1 RECLUTAMIENTO
  \rule{1\textwidth}{0.2mm}
  {\bf Algoritmo 1} Algoritmo de reclutamiento para el problema 1
  \hfill\break
  \rule{1\textwidth}{0.2mm}
  \hspace*{.2cm} {\bf Inicialmente hacer}
  \hfill\break
  1. {\bf if} yo soy un íder {\bf then}
  \hfill\break
  2. \hspace{0.5cm} parent $\leftarrow$ id
  \hfill\break
  3. \hspace{0.5cm} {\bf send}($<$ \textit{recluta} $>$) a ambos vecinos
  \hfill\break
  4. {\bf else}
  \hfill\break
  5. \hspace{0.5cm} parent $\leftarrow$ $\bot$
  \hfill\break
  6.{\bf end if}
  
  \hspace{0.2cm} {\bf Al recibir} recluta desde $p$ {\bf hacer:}
  \hfill\break
  7. {\bf if} parent $=$ $\bot$ {\bf then}
  \hfill\break
  8. \hspace{0.5cm} parent $\leftarrow$ $p$
  \hfill\break
  9. \hspace{0.5cm} {\bf send}($<$ \textit{recluta} $>$) a mi vecino que
  no es $p$
  \hfill\break
  10.{\bf end if}
  \hfill\break
  \rule{1\textwidth}{0.2mm}
  \hfill\break
  \ttfamily
  %Respoesta
  {\bf Solución:}
  
  Que el número de procesos sea par o no depende de $m$, pues una condición es
  que $k$ sea impar. El primer caso del algoritmo es cuando $k = 1$ y $m = 2$,
  de forma que $\{0,1\}$ son los procesos, donde por definición $\{0, (m-1)\cdot
  k = 1\}$ son los lideres, de forma que para el tiempo uno ya terminamos,
  con el arbol con raíz $0$ y otro con la raíz $1$. ¿Es esté el arbol, de todos los posibles, el que tiene menor altura?
  
  Sí, por que un árbol menor sería sin procesos, lo cúal las especificaciónes
  no permite. En general van a existir $m$ arboles.
  
  Nota: el árbol antes mencionado no es el único de altura cero,
  considere $k=1$ y $m=3$. Según lo dicho debe haber tres raíces:
  
  Tenemos los procesos $\{0,1,2\}$, de forma que dadas las instrucciónes
  $\{0, k=1, (m-1)\cdot k = 2\}$ son líderes, en el tiempo uno volvemos a
  terminar, pero al todos tener {\it id} nadie se anexa.
  Así, el árbol con menor tamaño posible es el que tiene un único elemento y
  altura cero.
  \newpage
  Así pues, tenemos $m$ arboles y $m\cdot k$ procesos, de forma que para
  cada árbol hay $k$ vertices. Como $k$ siempre es impar y cada capa mayor a
  cero tiene dos vertices, existen $\frac{k-1}{2}$ capas por árbol.
  De forma que el arbol con mayor vertices y altura esta determinado por $k$, ya
  que como mencionamos $m$ solo nos indica el número de arboles que tendran $k$
  elementos cada uno, entonces este sera cuando $m\in\mathbb{N}$ y $k$ tienda a
  infinito.

  Ejemplo: Cuando $m=2$ y $k=1001$, cada árbol tiene mil un elementos y altura
  quinientos.

  Ejemplo: Cuando $m=3$ y $k=1001$, cada árbol tiene mil un elementos y altura
  quinientos.
  
  \hfill\break
  Resultado de la ejecución con $k = 5$ y $m = 4$:
  \begin{figure}[h]
    \centering
    \def\svgwidth{15cm}
    \import{src/}{01.pdf_tex}
  \end{figure}
  \hfill\break
  Donde las aristas amarillas indican que procesos son seguidores de quien, y
  en las grises no hubo comunicación pero siguen existiendo.
  \hfill\break
  \rmfamily
  %02
\item Realice un análisis preciso de la complejidad de tiempo y la complejidad
  de mensajes de :
  \begin{itemize}
    %02 BRODCAST
  \item El algoritmo de broadcastTree.
    
    \hfill\break
    %Respoesta
    \ttfamily
    {\bf Solución:}

    Comenzamos enunciando el algoritmo:

    Suponemos la existencia de un árbol generador, $T$.

    % Algoritmo 2 BROADCASTREE
    \rule{1\textwidth}{0.2mm}
    {\bf Algoritmo 2} broadcastTree(ID, soyRaiz, {\it M}):
    \hfill\break
    \rule{1\textwidth}{0.2mm}
    1. PADRE, HIJOS, soyRaiz = soyRaiz
    
    \hfill\break
    \hspace*{.2cm} {\bf Inicialmente hacer}
    \hfill\break
    2. {\bf Si} soyRaiz {\bf entonces}
    \hfill\break
    3. \hspace{0.5cm} {\bf send}($<$ \textit{M} $>$) a todos los HIJOS
    \hfill\break
    
    \hspace{0.2cm} {\bf Al recibir $<${\it M}$>$ de} PADRE:
    \hfill\break
    4. \hspace{0.5cm} {\bf send}($<$ \textit{M} $>$) a todos los HIJOS
  \hfill\break
  \rule{1\textwidth}{0.2mm}
  \hfill\break
  En clase el profesor menciono que:
  \begin{itemize}
  \item Complejidad de los mensajes = $|V|-1$.
  \item Complejidad de tiempo = $Profundidad(T)$
  \end{itemize}
  Lo demostraremos a continuación, sea $G=\{V,E\}$ la gráfica que tiene al
  árbol generador $T$.
  \newpage
  Primero pensemos en la gráfica con un único vertice, llamemoslo $v$ si
  queremos que el algoritmo termine entonces debe ser raíz. Así:
  \begin{align*}
    u &\{\\
    &\text{Padre} : NULL\\
    &\text{Hijos} : \emptyset\\
    &\text{soyRaiz}: TRUE\\
    &\}
  \end{align*}
  Al $v$ ser el único proceso/vertice no le envía el mensaje a nadie, se cumple:
  \begin{align*}
    Mensajes(broadcastTree) &= |V|-1 = 1-1 = 0\\
    Tiempo(broadcastTree) &= Profundidad(T) = 0
  \end{align*}

  \hfill\break
  Generalizando:

  Mensajes:
  
  Los únicos vertices que no mandan mensajes son los que su conjunto HIJOS es
  vacío, es decir las hojas. Y el único vertice que no recibe mensaje es la
  raíz. Entonces los mensajes enviados son $V-1$.

  Tiempo:

  En el tiempo cero la raíz manda mensaje a todos los elementos del árbol
  en la capa/profundidad uno. (Ya mostramos el caso si es que la profundidad
  es cero). Decimos que para el tiempo $t$, con $t\geq 2$, los procesos en la
  capa/profundidad $t-1$ reciben y envian el mensaje.

  \hfill\break
  Sea $A=\{u|profundidad(u) = max(profundidad(v)) \forall v\in V\}$, entonces A es el conjunto
  de procesos de hojas con mayor profundidad. Sea $d=profundidad(T)$.

  Así, todos los demás procesos y en particular, en caso de existir,
  también las demás hojas que no comparten profundidad con cualquier $u_i\in A$,
  ya recibieron el mensaje. Si los procesos en la capa $d-1$ son hoja
  esos procesos ya terminaron, pero existe un vertice $v_i$ para cada
  proceso en $A$ tal que $d(v_i, u_i) = 1$ y, está de más decirlo pero lo
  repetimos, $profundidad(v_i) = d-1$. Así en el tiempo $d-1$
  recibieron y enviaron el mensaje, con lo que en $d$ terminamos.

  Concluimos:
  \begin{align*}
    Mensajes(broadcastTree) &= |V|-1 \\
    Tiempo(broadcastTree) &= Profundidad(T)
  \end{align*}
  \rmfamily
  \newpage
  %02 CONVERGECAST
\item El algoritmo de convergecast.
  
  \hfill\break
  %Respoesta
  \ttfamily
  {\bf Solución:}
      
  Comenzamos enunciando el algoritmo:
  
  Suponemos la existencia de un árbol generador, $T$.

  % Algoritmo 3
  \rule{1\textwidth}{0.2mm}
  {\bf Algoritmo 3} convergecast():
  \hfill\break
  \rule{1\textwidth}{0.2mm}
  1. PADRE, HIJOS, recibidos = 0
  
  \hfill\break
  \hspace*{.2cm} {\bf Inicialmente hacer}
  \hfill\break
  2. {\bf Si} |HIJOS| == 0 {\bf entonces}
  \hfill\break
  3. \hspace{0.5cm} {\bf send}($<$ \textit{ok} $>$) a PADRE
  \hfill\break
  
  \hspace{0.2cm} {\bf Al recibir $<${\it ok}$>$ de algún puerto} en HIJOS:
  \hfill\break
  4. recibidos++
  \hfill\break
  5. {\bf Si} recibidos == |HIJOS| {\bf entonces:}
  \hfill\break
  6. \hspace{1cm} {\bf send}($<$ \textit{ok} $>$) a PADRE
  \hfill\break
  \rule{1\textwidth}{0.2mm}

  \hfill\break
  % fin algoritmo
  En clase el profesor menciono que:
  \begin{itemize}
  \item Complejidad de los mensajes = $|V|-1$.
  \item Complejidad de tiempo = $Profundidad(T)$
  \end{itemize}
  Lo demostraremos a continuación, sea $G=\{V,E\}$ la gráfica que tiene al
  árbol generador $T$.

  \hfill\break
  Los algoritmos antes mencionados siempre hacian uso de la variable $soyRaiz$,
  si bien en este esta inicializada, no hacemos uso de ella. Lo que hace esté
  algoritmo es empezar con las hojas, que por definición su conjunto HIJOS
  es vacío, y de ahí subir. De nuevo comencemos viendo si para la gráfica
  con un unico proceso se cumple. Sea $u$ el único proceso, entonces en el
  tiempo cero el estado de $u$ es:
  \begin{align*}
    u &\{\\
    &\text{Padre} : NULL\\
    &\text{Hijos} : \emptyset\\
    &\text{soyRaiz}: TRUE\\
    &\text{recibidos}: 0\\
    &\}
  \end{align*}
  Su padre es NULL, si bien el algoritmo no especifica que en caso de ser
  raíz y $|HIJOS|== 0$ o $recibidos == |HIJOS|$ deberiamos terminar,
  entendemos que al ser padre vacio no se envía mensaje o más bien nadie
  lo recibe. En el tiempo $0$ se cumple que $|HIJOS|==0$,
  con lo que no hay una siguiente ejecución. Se cumple:
  \begin{align*}
    Mensajes(convergecast) &= |V|-1 = 1-1 = 0\\
    Tiempo(convergecast) &= Profundidad(T) = 0
  \end{align*}
  \newpage
  Generalicemos:
  
  Veremos casos donde hay mas de un vertice, pues el anterior ya lo mostramos.
  En el tiempo cero, los  primeros $k$ mensajes enviados son de las $k$ hojas
  del árbol $T$ que cumplen con la condición enumerada dos en el algoritmo.

  Con lo que en el tiempo uno, al menos un proceso recibe $r$ con
  $1\leq r\leq k$ mensajes, pues existe el caso que algún vertice tenga
  $k$ hojas como hijos.

  Existen al menos $m\geq 1$ procesos que para el tiempo dos se cumple
  $HIJOS==recibidos$, pues en caso de ser contrario el algoritmo no terminaria.
  Si $m$ es uno, llamemos a ese proceso $v$, entonces $v$ recibio $k$ mensajes.

  \hfill\break
  Sea $A=\{u|\ profundidad(u) == profundidad(T)\}$, entonces en $A$ están las
  hojas de mayor profundidad. Observese que no necesariamente todos los
  elementos de $A$ son los que comienzan a mandar mensaje, pues puede existir
  una hoja de menor profundidad que al ser hoja cumple con la condicion inicial.
  Sea $B=\{$hojas que no tienen profundidad igual a la profundiad del arbol$\}$.

  Si $profundiad(T) == 1$ ya terminamos, pues $|B| = 0$ y para los $|A|$
  procesos su padre era la raíz.

  Sea $profundiad(T) = d$, así $A=\{u_1, u_2, ..., u_n\}$. Entonces $n +|B|$
  procesos envian mensajes en el tiempo cero, no podemos asegurar nada de los
  $|B|$ mensajes, pero si podemos afirmar que para el tiempo uno
  los $n$ procesos llenan la capa $d-1$ del árbol, por que para 
  los vértices de esa capa solo es posible que tengan hijos en la profundidad
  del árbol o sean hojas. Así en el tiempo $t$ podemos asegurar que se llena
  la capa $d-t$. El algoritmo termina cuando se llena la capa cero. Garantizamos
  que la capa cero se llena en el tiempo $t$ tal que $d-t = 0$, que es en el
  tiempo $t=d._\square$

  \hfill\break
  La complejidad de los mensajes se sigue de lo antes dicho, cada proceso envía
  mensaje una única vez, primero los de la capa $d$  y termina hasta que la capa
  uno se reporta con la raíz.
  \newpage
  \rmfamily
  %02 broadCONVERGAST
\item El algoritmo de broadConvergecastTree.
  
  \hfill\break
  %Respoesta
  \ttfamily
  {\bf Solución:}
  
  Suponemos la existencia de un árbol generador, $T$.

  % Algoritmo 4
  \rule{1\textwidth}{0.2mm}
  {\bf Algoritmo 4} broadConvergecastTree(ID, soyRaiz):
  \hfill\break
  \rule{1\textwidth}{0.2mm}
  1. PADRE, HIJOS, soyRaiz = soyRaiz, vecinos = 0
  
  \hfill\break
  \hspace*{.2cm} {\bf Inicialmente hacer}
  \hfill\break
  2. {\bf Si} soyRaiz {\bf entonces}
  \hfill\break
  3. \hspace{0.5cm} {\bf send}($<$ \textit{START} $>$) a todos en HIJOS
  \hfill\break
  
  \hspace{0.2cm} {\bf Al recibir $<${\it START}$>$ de } PADRE:
  \hfill\break
  4. {\bf Si} |HIJOS| $\neq 0$ {\bf entonces:}
  \hfill\break
  5. \hspace{.5cm} {\bf send}$(<$ {\it START}$>)$ a todos en HIJOS
  \hfill\break
  6. {\bf Sino} 
  \hfill\break
  7. \hspace{1cm} {\bf send}($<$ \textit{ok} $>$) a PADRE
  \hfill\break

  \hspace{0.2cm} {\bf Al recibir $<${\it ok}$>$ de algún puerto} en HIJOS:
  \hfill\break
  8. vecinos++
  \hfill\break
  9. {\bf Si} vecinos == |HIJOS| entonces:
  \hfill\break
  10.\hspace{0.5cm} {\bf Si} soyRaiz entonces:
  \hfill\break
  11.\hspace{1cm} {\bf Reportar terminación}
  \hfill\break
  12.\hspace{0.5cm} {\bf Sino}
  \hfill\break
  13. \hspace{1cm}{\bf send}$(<${\it ok}$>)$ a PADRE
  \hfill\break
  \rule{1\textwidth}{0.2mm}

  \hfill\break
  % fin algoritmo
  \end{itemize}
  
  \hfill\break
  %03
  \rmfamily
\item ¿Se basan los algoritmos de broadcastTree y convergecast en el
  conocimiento acerca del número de nodos en el sistema?¿Por qué?
  
  \hfill\break
  %Respoesta
  \ttfamily
  {\bf Solución:}

  Al ser un sistema distribuido un proceso no esta al tanto de los HIJOS de
  otro proceso y no hay tal cual una varible global que nos indice el número
  total de vertices, pero si cada proceso no conociera el número de sus
  hijos convergecast no funcionaría, pues es lo que le permite identificar
  a un proceso que no esta esperando algún mensaje.

  El número de nodos en los arboles resulta ser la suma de los HIJOS de todos
  los vertices + 1(la raíz), esto por que no hay ciclos.
  
  \hfill\break
  Por otro lado broadcastTree solo utiliza al PADRE, si tuvieramos un conjunto
  que indique a los padres no podemos saber mucho acerca del total de procesos
  en el sistema (arbol no binario), de forma que broadcastTree no se basa en
  eso.
  \hfill\break
  \rmfamily
  % 04
\item Investiga y explica brevemente el concepto de time-to-live(TTL) usado en
  redes de computadoras y úsalo para modificar el algoritmo flooding visto en
  clase, de modo que un lider comunique un mensaje $M$ a los procesos a
  distancia a lo más $d$ del líder ($M$ y $d$ son entradas del algoritmo);
  todos los procesos a distancia mayor no deberian recibir $M$. Da un breve
  argumento que demuestre que tu algoritmo es correcto, y también haz un
  analísis de tiempo y número de mensajes.
  
  \hfill\break
  %Respoesta
  \ttfamily
  {\bf Solución:}

  Time to live o tiempo de vida, también conocido como hop limit o limite de
  saltos, es una herramienta que delimita el lapso de vida de información en
  una computadora o una red. TTL suele ser implementado con algún contador o
  con un sello, es decir una secuencia de caracteres (cadenas) o información
  decodificada, que en cuanto el contador llegue a su objetivo o el sello
  expire, la información es descartada o revalidad. En las redes de
  computadoras, TTL previene que un paquete circule indefinidamente.
  TTL también suele utilizarse en sistemas cache.

  \hfill\break
  Mostramos nuestro algoritmo:
  
  % Algoritmo 5
  \rule{1\textwidth}{0.2mm}
  {\bf Algoritmo 5} floodingTTL(ID, Lider, M, d):
  \hfill\break
  \rule{1\textwidth}{0.2mm}
  1. flag = {\bf false}
  
  \hfill\break
  \hspace*{.2cm} {\bf Inicialmente hacer}
  \hfill\break
  2. {\bf Si} ID == Lider {\bf entonces} // Soy el lider
  \hfill\break
  3. \hspace{0.5cm} flag = {\bf true}
  \hfill\break
  4. \hspace{0.5cm} {\bf send}(<M, 1>) por todos los puertos

  \hfill\break
  \hspace{0.2cm} {\bf Al recibir algún mensaje $<${\it M, cont}$>$ de algún
    puerto:}
  \hfill\break
  5. {\bf Si not} flag {\bf and} $cont < d$ {\bf entonces:}
  \hfill\break
  6. \hspace{1cm} flag = {\bf true}
  \hfill\break
  7. \hspace{1cm} {\bf send}(<M, cont+1>) por todos los puertos
  \hfill\break
  \rule{1\textwidth}{0.2mm}

  Primero pense en crear una variable $cont$, pero los demas procesos no tienen
  acceso a las variables de los démas, entonces en el caso del lider
  decidí enviar $M$ junto a $1$, que es la distancía de si mismo a que la envia,
  para el caso donde reciben $M$ y el entero $cont$, sí no habían enviado el
  mensaje previamente y el entero es menor al rango que todos reciben entonces
  el proceso envía el mensaje a todos sus puertos junto al contador incrementado
  en uno. La condición es menor y no menor o igual por que con el menor
  garantizo que llegara al sucesor del predecesor de $d$, es decir $d$. Y si
  hubiese sido menor o igual el mensaje llegaría a los distancia $d+1$.

  \hfill\break
  Recordemos que en flooding los mensajes rebotan, si bien los procesos que
  ya enviaron el mensaje solo lo hacen una vez, estos escuchan el mensaje
  $n$ veces, donde $n$ es el número de puertos a los que está conectado.

  Ahora no podemos garantizar que todo elemento reciba el mensaje, pues
  es solo para procesos $p$ tales que $d(raiz, p) \leq d$

  En la clase habiamos considerado las aristas como bidireccionales
  de forma que ahora no necesariamente se envian $2|E|$ mensajes. Considere
  el siguiente conjunto
  $$E' = \{uv|\  d(raiz, u) \leq d \land d(raiz, v)\leq d\}$$
  Así, si seguimos considerando bidireccionales las aristas se envian $2|E'|$
  mensajes.

  Nota: Para que se envien $2|E|$ aristas se debe cumplir que para cualquier
  vertice , $v$, $d(raiz, v) \leq d$.
  \hfill\break
  \rmfamily
  % 05
\item Generaliza el algoritmo convergecast para recolectar toda la información
  del sistema. Esto es, cuando el algoritmo termine, la raíz debería tener
  todas las entrdas de todos los procesos. Analiza la complejidad de bits,
  es decir, el total de bits que son enviados sobre los canales de
  comunicación. (\textit{hint: Cada mensaje de información puede tomar k bits}).
  
  \hfill\break
  %Respoesta
  \ttfamily
  {\bf Solución:}

  De nuevo suponemos existe el arbol generador, T.
  
  % Algoritmo 6
  \rule{1\textwidth}{0.2mm}
  {\bf Algoritmo 6} convergecastRecolecta(input, soyRaiz):
  \hfill\break
  \rule{1\textwidth}{0.2mm}
  1. PADRE, HIJOS, soyRaiz = soyRaiz, recibidos = 0, micadena = input
  
  \hfill\break
  \hspace*{.2cm} {\bf Inicialmente hacer}
  \hfill\break
  2. {\bf Si} |HIJOS| == 0 {\bf entonces}
  \hfill\break
  3. \hspace{0.5cm} {\bf send}($<$ \textit{ok, micadena} $>$) a PADRE
  \hfill\break
  
  \hspace{0.2cm} {\bf Al recibir $<${\it ok, cadena}$>$ de algún puerto}
  en HIJOS:
  \hfill\break
  4. recibidos++
  \hfill\break
  5. micadena = micadena{\it ++}'', ''{\it ++}cadena
  \hfill\break
  6. {\bf Si} recibidos == |HIJOS| {\bf entonces:}
  \hfill\break
  7. \hspace{0.5cm} {\bf Si} soyRaiz {\bf entonces:}
  \hfill\break
  8. \hspace{1cm} {\bf regresa}(micadena) // fin algoritmo
  \hfill\break
  9. \hspace{0.5cm} {\bf else}
  \hfill\break
  10. \hspace{1cm} {\bf send}($<$ \textit{ok, micadena} $>$) a PADRE
  \hfill\break
  \rule{1\textwidth}{0.2mm}
  Donde {\it ++} es la concatenación de cadenas.
  \hfill\break
  \rule{1\textwidth}{0.2mm}

  En el tiempo cero la variable micadena, inicializada con input, vale/contiene
  $k$ bits para todos los procesos. Los mensajes en el tiempo cero siguen partiendo de las hojas.

  Si tenemos un único proceso, en el tiempo cero acaba y el total de bits
  enviados es cero. 

  Si $profundidad(T)==1$ entonces los $|V|-1$ vertices envian en el tiempo
  cero su entrada de $k$ bits, con lo que el total de bits enviados es
  $(|V|-1)\cdot k$. Para el tiempo uno raíz concatena su cadena inicial con
  las $|V|-1$ y el algoritmo termina. Si consideramos el reporte final de
  la raíz circulan $(|V|)\cdot k$ bits.

  Supongamos $profundidad(T)=d$.
  
  Para cada mensaje que le llegue a cualquier proceso, este proceso concatena
  la cadena recibida con la cadena micadena, inicializada en tiempo cero con la
  entrada del proceso.
  Así en el tiempo uno cuando sea el turno de enviar la variable micadena
  para todo proceso en la capa $d-1$, esta pesara $(|HIJOS|+1)\cdot k$, de
  manera que en el tiempo $t+1$, se garantiza que todo proceso en la capa
  $d-t$ envió su micadena a su padre que pesa $(|HIJOS|+1)\cdot k$ más lo
  acumulado por cada hijo. 

  Entonces el total de bits enviados sobre los canales de comunicación, si
  consideramos el reporte final de la raíz, es la suma de todos los
  vertices de las capas $c_i$ con $i\in[0,1,...,d]$, que por definición es
  $|V|$. Si se considera que el algoritmo termina en cuanto la capa 1
  le reporta a la raíz entonces son los vertices en las capas $[1,...,d]$,
  que es $(|V|-1)\cdot k$ bits.
  

  \hfill\break
  \rmfamily
  % 06
\item
  \begin{enumerate}
    % a)
  \item Da un algoritmo distribuido para contar el número de vértices en
    un árbol enraízado T, iniciando en la raíz.
    \hfill\break
    %Respoesta
    \ttfamily
    {\bf Solución:}

    % Algoritmo 7
  \rule{1\textwidth}{0.2mm}
  {\bf Algoritmo 7} broadConvergecastTreeCuentaV(ID, soyRaiz):
  \hfill\break
  \rule{1\textwidth}{0.2mm}
  1. PADRE, HIJOS, soyRaiz = soyRaiz, vecinos = 0, vertices = 1
  
  \hfill\break
  \hspace*{.2cm} {\bf Inicialmente hacer}
  \hfill\break
  2. {\bf Si} soyRaiz {\bf entonces}
  \hfill\break
  3. \hspace{0.5cm} {\bf Si} |HIJOS == 0| {\bf entonces:}
  \hfill\break
  4. \hspace{1cm} {\bf regresa} 1
  \hfill\break
  5. \hspace{0.5cm} {\bf Sino}
  \hfill\break
  6. \hspace{1cm} {\bf send}($<$ \textit{START} $>$) a todos en HIJOS
  \hfill\break
  
  \hspace{0.2cm} {\bf Al recibir $<${\it START}$>$ de } PADRE:
  \hfill\break
  7. {\bf Si} |HIJOS| $\neq 0$ {\bf entonces:}
  \hfill\break
  8. \hspace{.5cm} {\bf send}$(<$ {\it START}$>)$ a todos en HIJOS
  \hfill\break
  9. {\bf Sino} 
  \hfill\break
  10. \hspace{1cm} {\bf send}($<$ \textit{ok, 1} $>$) a PADRE
  \hfill\break

  \hspace{0.2cm} {\bf Al recibir $<${\it ok, accum}$>$ de algún puerto} en HIJOS:
  \hfill\break
  11. vecinos++
  \hfill\break
  12. vertices = vertices + accum
  \hfill\break
  13. {\bf Si} vecinos == |HIJOS| entonces:
  \hfill\break
  14.\hspace{0.5cm} {\bf Si} soyRaiz entonces:
  \hfill\break
  15.\hspace{1cm} {\bf Regresa} vertices // fin
  \hfill\break
  16.\hspace{0.5cm} {\bf Sino}
  \hfill\break
  17. \hspace{1cm}{\bf send}$(<${\it ok, vertices}$>)$ a PADRE
  \hfill\break
  \rule{1\textwidth}{0.2mm}

  \hfill\break
  % fin algoritmo
  \rmfamily
  % b)
\item Extiende tu algoritmo para una gráfica arbitraria $G$.
  
  \hfill\break
  %Respoesta
  \ttfamily
  {\bf Solución:}

      % Algoritmo 8
  \rule{1\textwidth}{0.2mm}
  {\bf Algoritmo 8} broadConvergecastGraphCuentaV(ID, soyRaiz):
  \hfill\break
  \rule{1\textwidth}{0.2mm}
  \hspace{0.2cm} {\bf Ejecutar inicialmente:}
  \hfill\break
  1. buildSpanningTree(ID, root, <hola?>)
  \hfill\break
  \hspace{0.2cm}{\bf  Dado el árbol T obtenido en el paso uno:}
  \hfill\break
  2. broadConvergecastTreeCuentaV(ID, soyRaiz) // Algoritmo 7 de esta tarea.
  \hfill\break
  \rule{1\textwidth}{0.2mm}

  \hfill\break
  % fin algoritmo
  \end{enumerate}
  \newpage
  % 07
  \rmfamily
\item Da un algoritmo distribuido para contar el número de vértices en cada
  capa de un árbol enraízado T de forma separada. Analiza la complejidad
  de tiempo y la complejidad de mensajes de tu algoritmo.

  \hfill\break
  %Respoesta
  \ttfamily
  {\bf Solución:}

  \hfill\break    
\end{enumerate}
\end{document}
